\documentclass[10pt]{article}
\usepackage{geometry}
\geometry{a4paper, total={170mm,257mm}, left=20mm, top=20mm}

\usepackage[utf8]{inputenc}
\usepackage[T1]{fontenc}

\usepackage{amsmath}
\usepackage{amssymb}

\usepackage[]{algorithm2e}

\usepackage{hyperref}

\title{Norbert Health tests}
\author{Oussama Ennafii}

\begin{document}
    \maketitle

    \section*{Computer Vision}
        \subsection*{\# 1}
            In order to register the two images, one can make use of a standard procedure in 3D vision for finding a rigid transformation (RANSAC) from one image to the other.
            This supposes that both images do not suffer from distortion.
            This process relies on three steps:
            \begin{enumerate}
                \item Detect invariant interest points based on efficient methods such as SIFT, SURF or ORB.
                \item Match the interest points based on their descriptors.
                \item Use a RANSAC derived method to find the corresponding homography which will be robust to outlier matches.
            \end{enumerate}
            This is described in details in Alg.~\ref{alg::ransac_homography}.

            \begin{algorithm}[htb]
                \KwData{\texttt{lwir\_image}, \texttt{gray\_image}, \texttt{interest\_point\_type} \(\in \{SIFT, SURF, ORB\}\), \texttt{descriptor\_distance}, \texttt{min\_matches}}
                \KwResult{Registered LWIR image.}

                lwir\_points, lwir\_descriptors = compute\_interest\_points(\texttt{lwir\_image}, \texttt{interest\_point\_type})\;
                gray\_points, gray\_descriptors = compute\_interest\_points(\texttt{gray\_image}, \texttt{interest\_point\_type})\;

                matches = knn\_matches(lwir\_descriptors, gray\_descriptors, \texttt{descriptor\_distance})\;
                fitered\_matches = lowe\_ratio\_test(matches, ratio = 0.7)\;
                \If{\(\vert\)fitered\_matches\(\vert\) < \texttt{min\_matches}}{
                    raise runtime\_error("Insufficient matches to compute registration")\;
                }
                \(H\) = compute\_homography(lwir\_points, gray\_points, matches)\;
                registered\_lwir\_image = apply\_homography(\texttt{lwir\_image}, \(H\))\;
                \caption{\label{alg::ransac_homography} RANSAC based registration.}
            \end{algorithm}

            Since both images do not have the same domain the descriptors may not be well suited in this case.
            \cite{huo2011multilevel} proposes the use a restricted SURF or SIFT in scale to alleviate this problem.
            In order to solve this problem of descriptor domain mismatch, other approaches could be used.
            One way to do so would be to learn a useful distance function between these descriptors.
            Another way would rely on extracting edges (with Canny Edge detector for instance) and trying to match the corresponding edges (with Iterative Closest Point (ICP) for example).

            LWIR images once registered could be used to compute a thermal (heat) map for a patient for instance.
        
        \subsection*{\# 2}
            Check folder xxx

    \section*{Audio coughing detection}
        The goal 
    \bibliographystyle{plain}
    \bibliography{references}
\end{document}
